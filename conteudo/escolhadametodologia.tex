%Escolha da Metodologia
\section{Justificativa da Abordagem}
	\setlength{\parindent}{10ex}
	Para a definição da abordagem foram estudados o Processo Unificado e o \textit{(Scaled Agile Framework)}.

	\subsection{Processo Unificado}
	O Processo Unificado...
	\subsection{SAFe}
	O SAFe...
	\subsection{Resultados Obtidos}

	De acordo com estudos realizados sobre o Processo Unificado e o \textit{Scaled Agile FrameWork}, do contexto de negócio e das características dos \textit{stakeholders}, chegamos a algumas questões a serem respondidas. São Elas:
	\begin{itemize}
		\item Integração:
			\begin{itemize}
			\item O time de desenvolvimento poderá se encontrar com alta frequência?
			\item O cliente terá disponibilidade alta para encontros?
		\end{itemize}
		\item Time:
		\begin{itemize}
			\item O time mudará durante o desenvolvimento do projeto?
			\item O time possui experiência?
			\item O time possui forte integração?
		\end{itemize}

		\item Negócio:
		\begin{itemize}
			\item A estrutura organizacional da empresa é estavel?
			\item O cliente demanda formalidades?
			\item O sistema é crítico?
			\item Os requisitos do projeto mudarão com frequência?
			\item O cliente demanda entrega contínua de Software?
		\end{itemize}
	\end{itemize}
	\par
	A partir das perguntas levantas, foram respondidas, individualmente por cada membro da equipe de desenvolvimento, as perguntas, e chegou-se a conclusão que: \par O time apesar de não possuir tamanha experiência, estão motivados a trabalhar com desenvolvimento ágil, e o farão em reuniões frequentes e semanais, alêm de possuirem forte integração resultante de projetos passados. \par
	O cliente não demanda formalidades, apesar de necessitar de documentação, e, com a possibilidade de mudança nos requisitos, foi optado por contínua entrega de software e um contato próximo com o cliente.
	\par A partir do resultado obtido, foi gerado a seguinte tabela, e optado a abordagem adaptativa \textit{SAFe} para o desenvolvimento do projeto



	%Tabela da Escolha da Metodologia
	\begin{table}[hbpt]
	\huge
		\begin{adjustbox}{width=1.2\textwidth}
			\begin{tabular}{|l|l|c|c|p{10cm}|}
				\hline
				\textbf{Itens} & \textbf{Características} & \textbf{Tradicional} & \textbf{Ágil} & \textbf{Descrição} \\ \hline
				\multicolumn{ 1}{|l|}{\textbf{Interação}} & Reuniões - equipe de desenvolvimento &  & x & A equipe de desenvolvimento se reunirá com frequência \\ \cline{ 2- 5}
				\multicolumn{ 1}{|l|}{} & Encontro com cliente &  & x & A equipe de desenvolvimento manterá contato próximo ao cliente \\ \hline
				\multicolumn{ 1}{|l|}{\textbf{Time}} & Mudança de equipe de desenvolvimento &  & x & Não haverá mudanças na equipe de desenvolvimento \\ \cline{ 2- 5}
				\multicolumn{ 1}{|l|}{} & Experiência da equipe &  & x & A equipe possui experiência com desenvolvimento ágil de software \\ \cline{ 2- 5}
				\multicolumn{ 1}{|l|}{} & Equipe integrada &  & x & A equipe se conhece e já trabalhou junta em trabalhos anteriores \\ \hline
				\multicolumn{ 1}{|l|}{\textbf{Negócio}} & Requisitos mutáveis &  & x & Provável evolução do sistema após o fim da primeira etapa de projeto. \\ \cline{ 2- 5}
				\multicolumn{ 1}{|l|}{} & Documentação extensiva para manter o sistema &  & x & (?)Cliente não requer documentação formal/extensa(?) \\ \cline{ 2- 5}
				\multicolumn{ 1}{|l|}{} & Entregas parciais & x &  & (?)Não há necessidade de entregas parcias do software (?) \\ \cline{ 2- 5}
				\multicolumn{ 1}{|l|}{} & Projeto não é crítico &  & x & O projeto em desenvolvimento não é critico, não exigindo que todo o projeto seja elicitado e bem definido no inicio de seu desenvolvimento \\ \hline
			\end{tabular}
		\end{adjustbox}
		\caption{Escolha da Metodologia}
		\label{Escolha da Metodologia}
	\end{table}

	 


