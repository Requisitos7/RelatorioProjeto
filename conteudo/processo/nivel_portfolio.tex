\subsection{Atividades}
	\subsubsection{Nível de Portifólio}
	\begin{enumerate}[label = (\alph*)]
		\item \textbf{Realizar \textit{Workshop} com o cliente para compreensão do contexto:}
		\begin{itemize}
			\item \underline{Descrição}: Nessa atividade será realizado uma reunião com os clientes e envolvidos para a compreensão do negócio e levantamento inicial de requisitos do sistema. O entendimento das necessidades dos envolvidos deverá ser alcançado nessa atividade.
			\item \underline{Tarefas}:
			\begin{itemize}
				\item \textbf{Preparação para o \textit{workshop}:} Definir um local, participantes e o facilitador do \textit{workshop}.
				\item \textbf{Realização do \textit{workshop}:} O surgimento natural de idéias divergentes deve ser anotado e não ulgado. Posteriormente, deve-se entrar num consenso do contexto de negócio.
				\item \textbf{Definições  pós-\textit{workshop}:} Após entrar em um consenso, a equipe de desenvolvimento terá compreensão do que o cliente necessita e deve realizar um documento contendo as concluões acordadas no \textit{workshop}.
			\end{itemize}
			\item \underline{Artefato(s) de Entrada}: Documento de contexto do negócio.
			\item \underline{Artefato(s) de saída}: Relatório do Workshop
			\item \underline{Papéis dos Envolvidos}: \textit{Product Owner} (Equipe de Modelagem), Equipe de Desenvolvimento.
		\end{itemize}

		\item \textbf{Definir Tema de Investimento:}
		\begin{itemize}
			\item \underline{Descrição}: Nessa atividade, o time definirá o tema de investimento para a priorização dos investimentos da organização.
			\item \underline{Tarefas}:
			\begin{itemize}
				\item \textbf{Definir tema de investimento e posicionamento da organização:} Deve-se criar um documento para estabelecer o tema de investimento da empresa.
			\end{itemize}
			\item \underline{Artefato(s) de Entrada}: Nenhum.
			\item \underline{Artefato(s) de saída}: Tema de Investimento
			\item \underline{Papéis dos Envolvidos}: \textit{Product Owner} (Equipe de Modelagem), Equipe de Desenvolvimento.
		\end{itemize}

		\item \textbf{Identificar os Épicos:}
		\begin{itemize}
			\item \underline{Descrição}: Nessa atividade, serão elicitados os épicos de negócio com o cliente.
			\item \underline{Tarefas}:
			\begin{itemize}
				\item \textbf{Identificar épicos:} A partir do relatório do \textit{workshop}, deverão ser identificados os épicos.
				\item \textbf{Documentar épicos:} Definição dos épicos e criação do \textit{backlog} de portifólio na ferramenta de gerência para documentar os épicos definidos.
			\end{itemize}
			\item \underline{Artefato(s) de Entrada}: Relatório do \textit{workshop}.
			\item \underline{Artefato(s) de saída}: \textit{Backlog} de Portifólio
			\item \underline{Papéis dos Envolvidos}: \textit{Product Owner} (Equipe de Modelagem), Equipe de Desenvolvimento.
		\end{itemize}
	\end{enumerate}