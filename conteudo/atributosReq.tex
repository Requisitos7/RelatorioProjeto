\section{Atributos de Requisitos}
	Atributos de requisitos são propriedades dos mesmos e armazenam informacões adicionais. Para o desenvolvimento deste projeto, foram escolhidos os seguintes atributos:
	\begin{itemize}
		\item Prioridade
		\item Status
		\item Dificuldade
	\end{itemize}

	\subsection{Prioridade}
	O atributo \textbf{prioridade} indica a importância que as histórias de usuários têm para os \textit{stakeholders}. A prioridade da história de usuário pode ser definida em:

	\begin{table}[htbp]
		\centering
		\caption{Atributo de prioridade}
		\begin{tabular}{|l|l|}
			\hline
			\textbf{Atributo} & \textbf{Descrição} \\ \hline
			Alta & Resquisito que possui grande interesse para o cliente \\ \hline
			Média & Resquisito que possui elevado interesse para o cliente \\ \hline
			Baixa & Resquisito que possui pouco interesse para o cliente \\ \hline
			Indefinida & Prioridade indefinida \\ \hline
		\end{tabular}
	\label{Atributo de prioridade}
	\end{table}


	\subsection{Status}
	O atributo \textbf{status} indica a fase ou progresso atual do requisito. O status de um requisito pode ser definido em:
	\begin{table}[htbp]
		\centering
		\caption{Atributo de status}
		\begin{tabular}{|l|l|}
			\hline
			\textbf{Atributo} & \textbf{Descrição} \\ \hline
			ToDo & Resquisito identificado porem não definido \\ \hline
			In Progress & Resquisito difinido porem não completado \\ \hline
			To Verify & Requisito implementado porêm não verificado ou aceito \\ \hline
			Done & Requisitos completado \\ \hline
		\end{tabular}
	\label{Atributo de status}
	\end{table}


	\subsection{Dificuldade}
	O atributo \textbf{dfilculdade} indica o nível de esforço necessário para o desenvolvimento da história de usuário. A dificuldade de uma história de usuário pode ser definido em:

	\begin{table}[htbp]
		\centering
		\caption{Atributo de dificuldade}
		\begin{tabular}{|l|l|}
			\hline
			\textbf{Atributo} & \textbf{Descrição} \\ \hline
			Alta & Resquisito que possui grande dificuldade para ser implementado \\ \hline
			Média & Resquisito que possui elevada dificuldade para ser implementado \\ \hline
			Baixa & Resquisito que possui pouco dificuldade para ser implementado \\ \hline
			Indefinida & Dificuldade indefinida \\ \hline
		\end{tabular}
	\label{Atributo de dificuldade}
	\end{table}

	