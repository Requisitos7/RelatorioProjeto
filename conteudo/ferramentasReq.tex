\section{Ferramentas de Gerência de Requisitos}
	
	\subsection{Ferramentas Analisadas}
	
		Foram analisadas várias ferramentas de gerência de requisitos, dentre elas, ferramentas web com tempo limite de uso (dentre 7 a 1 mês) como: RequirementOne; SpiraTest; Jira; Visure Requirement; InteGREAT. Ferramentas web free, porem com recursos limitados, como a Rally e ferramentas onde é necessário a instalação de vários pacotes na máquina local, o que torna a sua instalação difícil e até mesmo não viável, como a IBM Rational DOORS, Axiom 4, codeBeamer e a enterprise Architect.

	\subsection{Ferramentas web com tempo máximo de uso}
	\begin{itemize}
		\item RequirementOne;
		\item SpiraTest;
		\item Jira;
		\item Visure Requirement;
		\item InteGREAT.
	\end{itemize}
		Todas as ferramentas acima foram descartadas pela equipe de desenvolvimento principalmente pelo seus curtos períodos de testes (média de 14 dias) e o custo para adquirição mensal.
		\par A ferramenta RequirementOne chegou a ser testada e aprovada pela equipe em primeira análise, porêm foi notado que ela não atende a todas as necessidades.

	\subsection{Ferramentas web free limitadas}
		Foi estudada e testada a ferramenta RALLY, desenvolvida para auxiliar desenvolvedores na criação de grandes projetos de abordagem ágil, com grande foco na metodologia lean e no SAFe, porêm sua versão free apenas permite registro e rastreabilidades de abstrações no nível de time (para pequenos projetos), deixando abstrações a nível de programa e portfólio apenas para contribuidores. Foi estudada também a ferramenta ReqView, é uma ferramenta mais simples que possibilita a rastreabilidade dos requisitos e a manter diferentes atributos.

	\subsection{Ferramentas locais}
		As ferramentas IBM Rational DOORs e Axiom 4 foram estudadas através de tutoriais e várias tentativas de instalações foram realizadas sem sucesso. Já a ferramenta codeBeamer foi instalada, utilizada e testada com sucesso, porêm apesar desta realizar toda uma rastreabilidade e gerência dos requisitos, não foi encontrado a possiblidade de se manter uma tabela dos atributos do mesmo. Por ultimo foi testada a ferramenta Enterprise Architect, ela é uma ferramenta muito boa para projetos desenvolvidos nas abordagens tradicionais, está sendo estudado a possibilidade de adequa-la às abordagens ágeis.

	\subsection{Escolha da Ferramenta}
		Para a escolha final da ferramenta o grupo está testando as ferramentas RALLY, codeBeamer, Enterprise Architect e a ReqView, onde será escolhida a que melhor atender as nossas necessidades.