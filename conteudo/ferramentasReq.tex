\section{Ferramentas de Gerência de Requisitos}
	
	\subsection{Ferramentas Analisadas}
		Foram analisadas várias ferramentas de gerência de requisitos, dentre elas, ferramentas web com tempo limite de uso (dentre 7 a 1 mês) como: RequirementOne; SpiraTest; Jira; Visure Requirement; InteGREAT. Ferramentas web free, porem com recursos limitados, como a Rally e o ReqView, e ferramentas onde é necessário a instalação de vários pacotes na máquina local, o que torna a sua instalação difícil ou até mesmo não viável, como a IBM Rational DOORS, Axiom 4, codeBeamer e a enterprise Architect.
		\par Para escolha da ferramenta, foi avaliado se a ferramenta possuía os seguites quesitos:
		\begin{itemize}
		\item possibilidade de abstração a nível de portfólio;
		\item possibilidade de abstração a nível de programa;
		\item possibilidade de abstração a nível de time;
		\item matriz de Rastreabilidade:
			\begin{itemize}
			\item rastreabilidade Horizontal;
			\item rastreabilidade Vertical;
			\end{itemize}
		\item tabela de Atributo de Requisitos:
			\begin{itemize}
			\item implementação de novos atributos;
			\end{itemize}
		\item controle de backlog;
		\end{itemize}
		
		\par
		Todas as ferramentas citadas acima que possuiam um tempo limitado de uso menor ou igual a um mês foram descartadas, apesar de tais ferramentas serem as mais faceis de serem usadas. As ferramentas IBM Rational DOORs e Axiom 4 foram estudadas através de tutoriais e várias tentativas de instalações foram realizadas sem sucesso, tanto no ambiente linux quanto no windows.
		\par
		As ferramentas ReqView e Enterprise Architect foram implementadas com êxito, porêm a reqview é muito simples e não atendia às necessidades do projeto, já a Enterprise Architect é um ferramenta especializada para projeto com abordagens tradicionais, sendo dificil a utilização para abordagem ágil, com isso ambos também foram descartadas, restando apenas as ferramentas Rally e a codeBeamer.\\

		A ferramenta RALLY foi desenvolvida para auxiliar desenvolvedores na criação de grandes projetos de abordagem ágil, com grande foco na metodologia lean e no SAFe. Sua versão free permite registro e rastreabilidades de abstrações no nível de time (para pequenos projetos), deixando abstrações a nível de programa e portfólio apenas para contribuidores, com isso chegamos a nossa escolha final da ferramenta, a codeBeamer.

	\subsection{Ferramenta Escolhida: codeBeamer}
		A ferramenta codeBeamer é uma ferramente de gerência de requisitos e de projetos gratuita, com limite a um projeto com duração de um ano.
		\\Ela permite o usuário montar qual tipo de rastreabilidade ele necessita a partir dos identificadores criados ou pré-existentes (histórias de usuário, features, epicos, temas, entre outros) permitindo assim uma rastreabilidade tanto na horizontal quanto na vertical.
		O codeBeamer permite ainda a criação de novos atributos, tornando possível a visualização de uma matriz de atributos de requisitos completa a partir da necessidade do usuário.
		\par A ferramente codeBeamer conseguiu atender a todos os quesitos exigidos para a implementação deste projeto, se tornando assim, a ferramente escolhida.
